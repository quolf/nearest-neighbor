\subsection{Modeling}
\label{results_models}
The first three experiments were replicated in two strains of mice (\ref{methods}). The identify of the S+ was varied across replications.  Observing similar results across strains and $S^+$ identities (Figure \ref{fig:something}), we pooled the data such that one component of the $S^+$ was designated as the reference component, and all other components in the $S^+$ or the test stimuli were labeled according to the difference in carbon chain length among their components relative to the reference (Figure \ref{fig:else}).  For example, for a $S^+$ consisting of 2-heptanol and 2-octanol, we designated 2-heptanol as the reference, $0$, and 2-octanol as $+1$.  Together the $S^+$ could be labeled $(0,+1)$.  A test stimulus of 2-octanol and 2-nonanol is then labeled $(+1,+2)$.  This allowed us to present data from using different $S+$ identities together on the same plot (Figure \ref{fig:else}).  

To account for the shape of the generalization gradients in each experiment, we developed a series of models that could be uniformly applied to each data set in order to account for  the similarity and overlap of the component odorant mixtures  To capture the data, a candidate model needs to reflect that component similarity is predictive of generalization, and that component overlap may be a special case of perfect similarity.  

We considered several possible candidate models (Figure X):