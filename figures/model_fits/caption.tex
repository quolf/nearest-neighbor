\label{fig:model_fits}
Model fits for the "Nearest Neighbor" model.  For each experiment in Fig. \ref{fig:results}, the data was normalized so that the first component of the $S+$ is defined to have a relative carbon chain length of 0.  Each stimulus is then described according to the carbon chain lengths of its components relative to the $S+$.  The resulting "pooled" experimental results are then averaged across mouse strains (dotted line). The solid line gives the prediction from the "Nearest Neighbor Model" for each experiment.  The same parameters are used across fits, except for the baseline response probability, which is allowed to vary by experiment.   