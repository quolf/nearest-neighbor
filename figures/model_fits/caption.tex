\label{fig:model_fits}
Model fits for the \textit{Nearest-Neighbor} model.  For each experiment in Fig. \ref{fig:results}, the data were normalized so that the first component of the $S+$ is defined to have a relative carbon chain length of 0.  Each stimulus is then described according to the carbon chain lengths of its components relative to the $S+$.  The resulting "pooled" experimental results are then averaged across mouse strains (dotted line). The solid line gives the prediction from the \textit{Nearest-Neighbor} model for each experiment.  The same parameters were used across fits to the three models (Fig. \ref{fig:cartoon}), except for the baseline response probability, which was allowed to vary by experiment.   