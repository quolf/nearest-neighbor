\label{fig:results}
Results of behavioral experiments. Animals were trained to discriminate the $S+$ (indicated by an open square) from a background (mineral oil) control. Black circles represent the novel testing stimuli ($Gen$) not experienced during training. The Y axis indicates the response probability to each testing and training odor. Error bars indicate the standard error of the mean.  Seperate lines indicate experiments on three different strains of mice. \textbf{A}) Exp. 1 Single-Linear:  X-axis shows carbon chain length of stimuli, where each stimulus was a secondary straight-chain alcohol, e.g. 2-heptanol.  \textbf{B})  Exp. 2 Double-Linear: Same as \textit{A}, except stimuli are binary mixtures of molecules at equal concentration.  As in \textit{A}, adjacent stimuli are 1-carbon chain length apart. \textbf{C}) Exp. 3 1-Shared: Similar to \textit{B}, except each binary mixture contains 2-heptanol in common. \textbf{D})  Exp. 4 1-Distant: Similar to \textit{B}, except each component in each test mixture is 1-carbon chain length away from the components in the $S+$. \textbf{E}) Exp. 5 Vapor Pressure Control: Test stimuli vary from the training stimulus by concentration, but are otherwise identical. Response probability is a linear function of concentration.  n=5 mice per strain for each experiment.  