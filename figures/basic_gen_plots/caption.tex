\label{fig:results}
Results of behavioral experiments. \textbf{A}) Probability of response during testing phase when animals were trained to discriminate the $S+$ (indicated by an open square) from a background control.  Test stimuli not experienced during training are indicated with closed circles.  Dashed vs dotted lines indicate experiment on two different strains (using a different $S+$ for each).  X-axis shows carbon chain length of stimuli, where each stimulus was a secondary straight-chain alcohol, e.g. 2-heptanol.  \textbf{B}) Same as \textit{A}, except stimuli are binary mixtures of molecules at equal concentration.  As in \textit{A}, adjacent stimuli are 1-carbon chain length apart. \textbf{C}) Similar to \textit{B}, except each binary mixture contains 2-heptanol in common. \textbf{D}) Similar to \textit{B}, except each component in each text mixture is 1-carbon chain length away from the components in the $S+$. \textbf{E}) A vapor pressure control, showing that animals response probability increases with vapor pressure, rather than learning to respond to the trained vapor pressure.  n=5 mice per strain for each experiment.  