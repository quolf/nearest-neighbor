\label{fig:results}
Results of behavioral experiments. The Y axis indicates the response probability to each testing and training odor. Error bars indicate the standard error of the mean. White squares indicate the training stimuli ($S+$ and black circles represent the novel testing stimuli ($Gen$) not experienced during training \textbf{A}) Exp. 1 Single-Linear: Animals were trained to discriminate the $S+$ (indicated by an open square) from a background (mineral oil) control.  Test stimuli not experienced during training are indicated with closed circles.  Dashed vs dotted lines indicate experiments on two different strains of mice (using a different $S+$ for each).  X-axis shows carbon chain length of stimuli, where each stimulus was a secondary straight-chain alcohol, e.g. 2-heptanol.  \textbf{B})  Exp. 2 Double-Linear: Same as \textit{A}, except stimuli are binary mixtures of molecules at equal concentration.  As in \textit{A}, adjacent stimuli are 1-carbon chain length apart. \textbf{C}) Exp. 3 1-Shared: Similar to \textit{B}, except each binary mixture contains 2-heptanol in common. \textbf{D})  Exp. 4 1-Distant: Similar to \textit{B}, except each component in each text mixture is 1-carbon chain length away from the components in the $S+$. \textbf{E}) Exp. 5 Vapor Pressure Control: Response probability increases with vapor pressure, rather than learning to respond to the trained vapor pressure.  n=5 mice per strain for each experiment.  