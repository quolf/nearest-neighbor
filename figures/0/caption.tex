\scriptsize Scalar: Each mixture gets one number.  For simplicity let's just say it is the average carbon chain length.  So a mixture of 7 and 9 gets an 8.  In this and all models we allow the relative weight to vary, so it could be 7.5 instead.  In any case, it is one number.  It gets compared against the one number from the other (test) mixture.  

All to All: Each mixture gets one number per component.  When comparing a mixture of 6 and 8 to a mixture of 8 and 10, we compared 6 to 8, 6 to 10, 8 to 8, and 8 to 10.  In other words, all to all comparison.  Then we add these up.  

Nearest Neighbor: This is just like all-to-all, except for each component in the trained mixture (i.e. in the CS+), we only make one comparison to a component in the tested mixture.  So if we trained on 6,8 and test on 8,10, instead of comparing all to all, we only compare 6 to 8 (the nearest thing to 6 in the test mixture), and 8 to 8 (likewise).  This model basically says that the dissimilar things are not so important.  It's more a matter of how many similar things they are (and how similar) as a fraction of the total number of components.  