\label{fig:cartoon}
Three candidate models for olfactory generalization.  Circles represents components in a binary mixture, with different colors representing different components.  Arrows show perceptual representations, with similarity computed in three ways. (\textbf{A}) in the \textit{Mean} model, each mixture is reduced to one scalar value, e.g. average carbon chain length.  The scalar value for the trained mixture is compared against the scalar value for the test mixture, with their similarity determining the probability of response. (\textbf{B}) In the \textit{All-to-All} model, each mixture gets one value per component, and comparison between values is "all to all" between trained and tested components.  All comparisons are averaged to determine response probability. (\textbf{C}) In the \textit{Nearest-Neighbor} model, each mixture gets one value per component, but comparison only occurs between a trained component and the "nearest" component in the test mixture.  