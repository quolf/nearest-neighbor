\subsection*{Models}
\label{sec:methods_models}

We modeled the data by considering only the carbon chain length of the molecules composing each stimulus.  
The computation of the animal's response probability to each test stimulus was given by first computing: 
\begin{equation}
f(S+_i,test_j) = f_1(S+_i) * f_2(S+_i,test_j)
\end{equation}
with each function argument the carbon chain length of an $S+$ component $i$ or a test stimulus component $j$, where:
\begin{equation}
f_1(x_i) = exp(-a*x_i) \\
f_2(x_i,x_j) = exp(-b*|x_i-x_j|)
\end{equation}
with $a$ and $b$ free parameters, obtained by maximum likelihood estimation (see below).  
$f_1$ represents the salience of an $S+$ component, and the functional form of $f_1$ is inspired by the vapor pressure dependence of carbon chain length, which is one component of perceived intensity.  $f_2$ represents the generalization gradient across carbon chain length.  

We then considered 3 classes of models, varying in computational complexity, i.e. the number of computations of $f$ across stimulus components.  
\begin{itemize}
\item \textit{Scalar}: Each stimulus was represented by the mean carbon chain length of its components.  The probability of response ($p_r$) was a function of the difference between the mean carbon chain length of the $S+$ and that of the test stimulus, i.e.:
\begin{equation}
p_r = p_{r0} * f(S+_mean,test_mean)
\end{equation}
\item \textit{All-to-All}: Each stimulus was represented by an array of length N containing the carbon chain lengths of the components, e.g. an array of length 2 for a binary mixture. $p_r$ was a function of the difference between each element of the $S+$ array and each element of the test stimulus array, i.e. the difference in carbon chain lengths between each $S+$ component and each test stimulus component.  
\begin{equation}
p_r = p_{r0} * \frac{1}{N^2}\sum{i=1}{N}\sum{j=1}{N}f(S+_i,test_j)
\end{equation}
\item \texit{Nearest Neighbor}: Each stimulus was represented as in \textit{All-to-All}, but $p_r$ was a function only of the difference between each element of the $S+$ array and the single nearest (most similar in carbon chain length) element of the test stimulus array, with index i'.  
\begin{equation}
p_r = p_{r0} * \frac{1}{N}\sum{i=1}{N}{N}f(S+_i,test_i')
\end{equation}
\end{itemize}



For the \textit{scalar| model, f

\item Fit details
\item Goodness of fit statistics
\item Equations for all models
\item References to code
