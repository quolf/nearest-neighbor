\subsection*{Models}
\label{sec:methods_models}

We modeled the data by considering only the carbon chain length of the molecules composing each stimulus.  
We considered 3 classes of models, varying in computational complexity.  
\begin{itemize}
\item \textit{Scalar}: Each stimulus was represented by the mean carbon chain length of its components.  The probability of response ($p_r$) was a function of the difference between the mean carbon chain length of the $S+$ and that of the test stimulus.  
\item \textit{All-to-All}: Each stimulus was represented by an array of the carbon chain lengths of the components, e.g. an array of length 2 for a binary mixture. $p_r$ was a function of the difference between each element of the $S+$ array and each element of the test stimulus array, i.e. the difference in carbon chain lengths between each $S+$ component and each test stimulus component.  
\item \texit{Nearest Neighbor}: Each stimulus was represented as in \textit{All-to-All), but $p_r$ was a function only of the difference between each element of the $S+$ array and the single nearest (most similar in carbon chain length) element of the test stimulus array.  

In each model class above, the computation in each stimulus comparison was given by: 
\begin{equation}
f(S+_i) * g(S+_i,test_j)
\end{equation}
with each function argument the carobn chain length of an $S+$ component $i$ or a test stimulus component $j$, where:
\begin{equation}
f(x)
\end{equation}

\item Fit details
\item Goodness of fit statistics
\item Equations for all models
\item References to code
\end{itemize}
