\section*{Methods}
\label{sec:methods}
\subsection*{Experiments}

\subsubsection*{Subjects}
\label{sec:methods_subjects}
Male mice (N=XXXX, PND 30-60) served as subjects. The mice were single-housed in a colony room with a reverse 12:12 hr light cycle (dark 7 AM to 7 PM). Three strains of mice were trained and tested: C57BL/6 (C57); an outbred strain on mixed background (MEK) with deficiencies in the MEK2 gene, implicated in ERK/MAP kinase signaling cascade; and an inbred strain of mice expressing channel rhodopsin on acetylcholine transferase positive neurons (ChR:ChAT). The MEK2 deficient mice were acquired for initial experiment piloting, and the mutation has no known behavioral or health effects, possibly due to compensation by the closely related kinase MEK1 \cite{12832465}. The ChR:ChAT mice were established by by crossing a ChAT:Cre expressing mouse strain with a Cre-dependent ChR2 line known as “Ai32” (Rosa-CAG-loxpSTOPloxp-hChR2H134R-EYFP-WPRE). All strains of mice were acquired from an in-house breeding program. All mice were water restricted, beginning seven days prior to the experiment. During water restriction, each mouse received 1 mL of water in their home cage at 4:30 pm each day, in addition to any liquid they earned in the experimental session. Experimental sessions occurred between 9 AM and 4 PM, seven days a week. All animal use and experimental procedures conformed to guidelines established by the National Institutes of Health (NIH) Guide for the Care and Use of Laboratory Animals and the Arizona State University Institutional Animal Care and Use Committee.
  