\section*{Methods}
\label{sec:methods}
\subsection*{Experiments}

\subsubsection*{Subjects}
\label{sec:methods_subjects}
Male mice (PND 30-60) served as subjects. The mice were single housed in a colony room with a reverse 12:12 hr light cycle (dark 7 AM to 7 PM). Two strains of mice were trained and tested: C57BL/6 (C57) and an outbred strain on mixed background (MEK) with deficiencies in the MEK2 gene, implicated in ERK/MAP kinase signaling cascade. The MEK2 deficient mice were acquired for initial experiment piloting, and the mutation has no known behavioral or health effects, nor were any observed, presumably due to compensation by the closely related kinase MEK1 \cite{12832465}. Both strains of mice were acquired from an in-house breeding program. All mice were water restricted, beginning seven days prior to the experiment. During water restriction, each mouse received 1 mL of water in their home cage at 4:30 pm each day, in addition to any liquid they earned in the experimental session. Experimental sessions occurred between 9 AM and 4 PM, seven days a week.

\subsubsection*{Apparatus}
\label{sec:methods_apparatus}
Experiments were conducted in a ``Slotnick style'' 4-channel pinch-valve olfactometer (described in \cite{18428626}) and operant chamber purchased from Knosys Systems. All odorants were diluted to $0.1\%$ in mineral oil, except as noted in the text and described in the Experiments section below.

A set of stereo computer speakers in the cabinet were used to project background noise during the experiment.  Experimental events were controlled and recorded by software custom written in Python. The reinforcer was approximately 0.006 mL of a chocolate milk (Shamrock Farms; Phoenix, AZ) and water mixture ($\frac{1}{3}$ milk by volume) combined with the sounding of a brief (0.3 s) 9 kHz buzzing tone at 65 dB. Subjects were given approximately 0.5 mL of the milk mixture 24 hours prior to the first experimental session to familiarize them with the reward they would experience in behavioral experiments.  

\subsubsection*{Procedure}
\label{sec:methods_procedure}
The odorant discrimination training and generalization testing protocol was a modified version of the Go/No Go task used by Slotnick and colleagues \cite{18428626}. Each experiment consisted of four phases: Pretraining, Odorant discrimination (full reinforcement), Odorant discrimination (partial reinforcement), and Generalization Testing. During pretraining, subjects were trained to lick the liquid delivery tube and keep their head in the odor tube for several seconds. No odor was presented during pre-training. During odorant discrimination (full reinforcement), subjects were trained to lick the reinforcer spout for chocolate milk when one odorant (the $S+$) was presented, but not when mineral oil (the $S-$) was presented; correct responses to all $S+$ trials were reinforced. Because generalization trials with the novel stimuli must be conducted in the absence of reinforcement to prevent training the animal to respond to the novel stimuli, subjects were trained to persist in responding during low rates of reinforcement in Odorant discrimination (partial reinforcement). In that phase, the frequency of reinforcement on $S+$ trials was gradually reduced from $100\%$ to $20\%$. During generalization testing, each subject's responses to three novel, unreinforced odorants were examined.

\subsubsection*{Pretraining and Odor Discrimination}
\label{sec:methods_pretraining}
Pretraining and Odor discrimination training proceeded according to Slotnick and Restrepo (1994). Discrimination training consisted of the standard Go/No Go task in which subjects were trained to lick in the presence of one odor, the $S+$, and not in the presence of another odor, the $S-$.  Each odorant discrimination sessions began with a ``warm up'' block of 10 trials in which all trials were $S+$ in order to establish robust responding to the $S+$  (Nevin, 1998, Shahan 2014).  The remainder of the session consisted of 9 blocks of 10 $S+$ trials and 10 $S-$ trials, ordered pseudo-randomly so that no more than 3 trials of the same type ($S+$ or $S-$) occurred consecutively within a block. Once a subject responded with at least $80\%$ correct for two consecutive sessions, we advanced the protocol to partial reinforcement.

\subsubsection*{Odorant discrimination (partial reinforcement)}
\label{sec:methods_discrimination}
This phase was identical to odorant discrimination, except the probability of reinforcement on $S+$ trials after the ``warmup'' block was gradually reduced over four sessions. During the first session, $80\%$ of $S+$ trials resulted in reinforcement for correct responding. The following three sessions reduced the frequency of reinforcement on $S+$ trials to $60\%$, $40\%$ and then $20\%$. The next session after $20\%$ reinforcement was Generalization Testing.

\subsubsection*{Generalization testing}
\label{sec:methods_training}
During generalization testing, the propensity of a subject to respond to each of the three generalization odorants ($G1$, $G2$, and $G3$) relative to the $S+$ was tested. Generalization testing consisted of ``warm up'' block, as per odorant discrimination, and four generalization blocks. Generalization blocks consisted of eight trials – two trials with each generalization odorant ($G1$, $G2$, and $G3$), and two trials with the $S+$. Responding to the generalization odorants was never reinforced. However, to prevent response extinction, responding on one of the two $S+$ trials in each block was reinforced, while the other was not. In generalization blocks, trial order was determined pseudorandomly, in that all four types of trials ($S+$, $G1$, $G2$, and $G3$) must have occurred before a second trial of the same type occurred.

\subsubsection*{Experiment types}
\label{sec:methods_experiment_types}
Five different experiments were conducted across both strains of mice. Due to the availability of the mice, MEK2 were only trained and tested with experiments 1-3, while C57 was trained and tested with experiments 1-5.  Table \ref{tab:design} outlines the odorants used in each experiment.

\begin{enumerate}
\item \textit{Single-Linear}. Each stimulus (S+ and G1, G2, and G3) was a simple mixture of a single [2-n alcohol] diluted to $0.1\%$ concentration by volume in mineral oil (Sigma-Aldrich, M3516-1L). This experiment tested generalization to novel stimuli which differed from $S+$ by their carbon chain length.

\item \textit{Double-Linear}. Each stimulus was a binary mixture of two odorants [2-n alcohols], where each odorant in the mixture only differed from each other by a single carbon chain. Each odorant was diluted to $0.05\%$ concentration by volume in mineral oil. For example, the $S+$ for MEK2 was a mixture of $0.05\%$ 2-Heptanol, $0.05\%$ 2-Octanol, and $99.9\%$ mineral oil.  

\item \textit{Double 1-Shared}. Each stimulus was a binary mixture of 2 odorants [2-n alcohols], as diluted as in experiment \textit{2}. However, each of the generalization stimuli shared with the $S+$ one component (2-Heptanol) and differed from the $S+$ by one component. 

\item \textit{Double 1-Distant}. Each stimulus was a mixture of 2 odorants, diluted as in experiment \textit{2}. However, the generalization stimuli shared no overlapping components with the $S+$, but had identical or similar average carbon chain lengths. For example, the $S+$ was a mixture 2-heptanol and 2-nonanol, with an average chain length of 8, while $G1$ was a mixture of 2-Hexanol and 2-decanol, also with an average carbon chain length of 8.

\item \textit{Vapor Pressure Control}. When the dilution by volume remains constant across the single 2-n alcohols within our series, the partial vapor pressure decreases approximately 3.3 fold. To ensure that the results in experiments \textit{1}-\textit{4} could not be explained by simple differences in partial vapor pressures, experiment \textit{5} trained 2-Heptanol at $1.0\%$ dilution as the $S+$, and tested 2-Heptanol at $0.01\%$, $0.1\%$ and $5.0\%$ as the generalization stimuli.

\end{enumerate}
