\section*{Methods}
\label{sec:methods}
\subsection*{Experiments}

\subsubsection*{Subjects}
\label{sec:methods_subjects}
Male mice (PND 30-60) served as subjects. The mice were single-housed in a colony room with a reverse 12:12 hr light cycle (dark 7 AM to 7 PM). Two strains of mice were trained and tested: C57BL/6 (C57) and an outbred strain on mixed background (MEK) with deficiencies in the MEK2 gene, implicated in ERK/MAP kinase signaling cascade. The MEK2 deficient mice were acquired for initial experiment piloting, and the mutation has no known behavioral or health effects, nor were any observed, presumably due to compensation by the closely related kinase MEK1 \cite{12832465}. Both strains of mice were acquired from an in-house breeding program. All mice were water restricted, beginning seven days prior to the experiment. During water restriction, each mouse received 1 mL of water in their home cage at 4:30 pm each day, in addition to any liquid they earned in the experimental session. Experimental sessions occurred between 9 AM and 4 PM, seven days a week.

\subsubsection*{Apparatus}
\label{sec:methods_apparatus}
Experiments were conducted in a ``Slotnick-style'' 4-channel pinch-valve olfactometer (described in \cite{18428626}) and operant chamber purchased from Knosys Systems. All odorants were diluted to $0.1\%$ in mineral oil `blank', except as noted in the text and described in the Experiments section below.

Experimental events were controlled and recorded by software custom written in Python. The reinforcer was approximately 0.006 mL of chocolate milk (Shamrock Farms; Phoenix, AZ) diluted in water ($\frac{1}{3}$ milk by volume) combined with the sounding of a brief (0.3 s) 9 kHz buzzing tone at 65 dB from a set of stereo computer speakers. Subjects were given approximately 0.5 mL of the milk mixture 24 hours prior to the first experimental session to familiarize them with the reward they would experience in behavioral experiments.  

