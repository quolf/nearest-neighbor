\section{Results}

\subsection{Experiments}
In order to understand the relative influence of component overlap vs component similarity, we conducted five olfactory generalization experiments with binary mixtures of straight-chain aliphatic alcohols.  In each experiment, we trained mice to associate a single mixture ($S+$) with liquid reward and then tested their conditioned responses during exposure to the $S+$ or to related mixtures (\ref{Methods}).  Across experiments, the relationship between the $S+$ test stimuli was varied to determine what features of the $S+$ were most effectively generalized during testing.  

In the first experiment (``Single''), the $S+$ and the test stimuli were not mixtures, but single components -- since there is no possible overlap whatsoever between stimuli (i.e. no shared components), so only similarity to the $S+$ varies across the test stimuli.  These test stimuli has carbon chain lengths varying from 2 shorter to 2 longer than the $S+$  The generalization we observed (Fig. \ref{fig:something}a) indicated that similarity decreases with increasing carbon chain length difference between the S+ and the test stimuli, as reported previously \cite{}.  In order to characterize the degree of generalization, we used the XXXX index (\ref{methods}), for which a value of 1 reflects perfect generalization (the same response to every test stimulus as to the $S+$) and for which a value of 0 reflects no generalization (response only to the $S+$).  In this experiment the $XXXX index was XXXX \pm XXXX$.  

In the second experiment (``Double''), the $S+$ and the test stimuli were binary mixtures, with each mixture containing molecules adjacent in carbon chain length.  Each test stimulus was ``shifted'' from the $S+$ similar to experiment 1, except with both components shifted together.  Generalization similar to, but shallower than, experiment 1 was observed (Fig. \ref{fig:something}b).  This shallower generalization was reflected in an $XXXX index of XXXX \pm XXXX$.  

In the third experiment (``1-Shared''), the $S+$ and the test stimuli were binary mixtures differing in exactly one component, with one component in common.  The differing component varied systematically in carbon chain length.  In this experiment, generalization was almost totally flat (Fig. \ref{fig:something}c, $XXXX index = XXXX \pm XXXX$, indicating that the identity of the differing component was almost irrelevant.  

In the fourth experiment (``1-Distant''), the test stimuli had no overlap with the $S+$, but each component was either 1 carbon shorter or 1 carbon longer than a corresponding component in the $S+$.  Only a low level of generalization was observed (Fig. \ref{fig:something}d, $XXXX index = XXXX \pm XXXX$), similar to experiment 1.  

In the fifth experiment (``Vapor Pressure Control'') the test stimuli were identical to the $S+$, except they all varied in concentration.  Since alcohols of lower molecular weight tend to have higher vapor pressures (approximately 3-fold decrease with each additional carbon), we wanted to ensure that generalization was not occurring to vapor pressure rather than molecular identity.  We used the same molecule for the $S+$ and all test stimuli, and varied only concentration.  We found that response rate was linear in concentration, with no deviation reflecting a specific preference for the concentration of the $S+$.  This indicated that vapor pressure was not a variable to which the mice were specifically attuned.  