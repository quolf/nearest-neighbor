\section*{Results}
\label{results}
\subsection*{Experiments}
\label{results_experiments}
In the first experiment (``Single-Linear''; Table \ref{tab:design}i) we tested the direct physiochemical (molecular) similarity between the $S+$ and each of the test stimuli (G1-G3). 
The results (Fig. \ref{fig:results}a) indicated that generalization responses decrease with increasing carbon chain length difference between the $S+$ and the test stimuli, as reported previously for both mammals and insects \cite{18810459}\cite{24488965}.  

In the second experiment (``Double-Linear''; Table \ref{tab:design}ii), the $S^+$ and the test stimuli were a monotonic series of binary mixtures.
Generalization to the test stimuli occurred as a function of the degree of overlap of the mixtures with the $S+$. 
However, the generalization gradient was shallower than in experiment 1 (Fig. \ref{fig:results}b). 
The test stimuli that contained one component in common with the S+ elicited higher responses than the test stimuli closest to the S+ in experiment 1. 
Some generalization asymmetry was observed in one mouse line, but this was not significant after correcting for multiple comparisons.   

In the third experiment (``1-Shared''; Table \ref{tab:design}iii), the $S^+$ and the test stimuli were binary mixtures that differed in exactly one component -- that component varied systematically in carbon chain length.  
In this experiment, generalization was almost totally flat (Fig. \ref{fig:results}c), indicating that the identity of the differing component was almost irrelevant.  
The animals appeared to generalize to the common component.  

In the fourth experiment (``1-Distant''; Table \ref{tab:design}iv), the two components of the test stimuli were distinct from those of the $S^+$, but they spanned a similar range of carbon chain lengths. 
Only a low level of generalization was observed (Fig. \ref{fig:results}d), and it was similar to the generalization observed in the first experiment, Single-Linear.  

In the fifth experiment (``Vapor Pressure Control''; Table \ref{tab:design}v) the test stimuli were identical to the $S^+$ (2-heptanol), except they all varied in concentration.  
Alcohols of lower molecular weight tend to have higher vapor pressures.  
Therefore, in experiments 1-4 generalization could have been driven primarily by similarity to the vapor pressure of the $S^+$.  
We therefore performed an experiment to control for vapor pressure by using a single odorant (2-heptanol) for the $S^+$ and all test stimuli; the test stimuli then differed from the $S+$ only in concentration.  
We found that response rate was approximately linear in concentration, with no deviation reflecting a specific preference for the concentration of the $S^+$.  
This indicated that vapor pressure was not the principal quality that drove generalization responses in experiments 1-4.  