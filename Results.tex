\section{Results}

In order to understand the relative influence of component overlap vs component similarity, we conducted five olfactory generalization experiments with binary mixtures of straight-chain aliphatic alcohols.  In each experiment, we trained mice to associate a single mixture (S+) with liquid reward and then tested their conditioned responses during exposure to the S+ or to related mixtures (Methods).  Across experiments, the relationship between the S+ test stimuli was varied to determine what features of the S+ were most effectively generalized during testing.  

In the first experiment (“Single”), the S+ and the test stimuli were not mixtures at all, but single components -- since there is no possible overlap whatsoever between stimuli (i.e. no shared components), so only similarity to the S+ varies across the test stimuli.  These test stimuli has carbon chain lengths varying from 2 shorter to 2 longer than the S+  The generalization we observed (Fig. 1a) indicated that similarity decreases with increasing carbon chain length difference between the S+ and the test stimuli, as reported previously [refs].  In order to characterize the degree of generalization, we used the XXXX index (Methods), for which a value of 1 reflects perfect generalization (the same response to every test stimulus as to the S+) and for which a value of 0 reflects no generalization (response only to the S+).  In this experiment the XXXX index was XXXX +/- XXXX.  

In the second experiment (“Double), the S+ and the test stimuli were binary mixtures, with each mixture containing molecules adjacent in carbon chain length.  Each test stimulus was “shifted” from the S+ similar to experiment 1, except with both components shifted together.  Generalization similar to, but shallower than, experiment 1 was observed (Fig. 1b).  This shallower generalization was reflected in an XXXX index of XXXX +/- XXXX.  

In the third experiment (“1-Shared”), the S+ and the test stimuli were binary mixtures differing in exactly one component, with one component in common.  The differing component varied systematically in carbon chain length.  In this experiment, generalization was almost totally flat (Fig. 1c, XXXX index = XXXX +/- XXXX), indicating that the identity of the differing component was almost irrelevant.  

In the fourth experiment (“1-Distant”), the test stimuli had no overlap with the S+, but each component was either 1 carbon shorter or 1 carbon longer than a corresponding component in the S+.  Only a low level of generalization was observed (Fig. 1d, XXXX index = XXXX +/- XXXX), similar to experiment 1.  

In thethe fifth experiment (“Vapor Pressure Control”) the test stimuli were identical to the S+, except they all varied in concentration.l.  Since alcohols of lower molecular weight tend to have higher vapor pressures (approximately 3 fold decrease with each additional carbon), we wanted to ensure that generalization was not occurring to vapor pressure rather than molecular identity.  We used the same molecule for the S+ and all test stimuli, and varied only concentration.  We found that response rate was linear in concentration, with no deviation reflecting a specific preference for the concentration of the S+.  This indicated that vapor pressure was not a variable to which the mice were specifically attuned.  

Models of Odor Similarity

The first threeexperiments were replicated in two strains of mice (Methods), and we also used the opportunity to vary the identity of the S+ across replications.  Observing similar results across strains and S+ identities (Figure 1), we pooled the data such that one component of the S+ was designated as the reference component, and all other components in the S+ or the test stimuli were labeled according to the difference in carbon chain length among their components relative to the reference (Figure 2).  For example, for a S+ consisting of 2-heptanol and 2-octanol, we designated 2-heptanol as the reference, 0, and 2-octanol as +1.  Together the S+ could be labeled (0,+1).  A test stimulus of 2-octanol and 2-nonanol is then labeled (+1,+2).  This allowed us to present data from using different S+ identities together on the same plot (Figure. 2).  

To account for the shape of the generalization gradients in each experiment, we developed a series of models that could be uniformly applied to each data set in order to account for  the similarity and overlap of the component odorant mixtures  To capture the data, a candidate model needs to reflect that component similarity is predictive of generalization, and that component overlap may be a special case of perfect similarity.  

We considered several possible candidate models:

Model 1 - Scalar. In this model,, each mixture is represented by a scalar function of the carbon chain lengths of the mixture’s components.  Then some function of those scalars predicts generalization across the mixtures.  In this model different mixtures are treated identically, if they map to the same scalar.  For example, a mixture of 2-heptanol and 2-nonanol might be treated identically to solution of only 2-octanol (a midpoint in carbon chain length).  [LET’S PUT THE  MATHEMATICAL EQUATIONS IN]
Pr(Resp|Odorant) =  ? 

Model 2 - All to all.
At the other extreme, in the most complex model (“All to all”), mixtures are compared by computing a function of each component in the S+ to each component in the test mixture -- for N-component mixtures this means N2 computations (N2 = 4 since these are binary mixtures) -- with functions added to predict generalization.  In this model each component “interacts” with each other component, and any deviation in any combination of components across mixtures is potentially predictive of non-generalization.  
[STILL NOT SURE HOW THIS WORKS EXACTLY]

Model 3 - Nearest Neighbor.
This model takes each component of the S+, and compares it only to the “nearest” component in the test mixture, with nearest determined by carbon chain length.  Since there is only one function for each S+ component (a function of the component and its nearest neighbor in the test mixture), there are N computations (N = 2 here).  

Model 4 - Nearest Neighbor with Vapor Pressure correction. 
[additional description and equation here] 
Because vapor pressure increases with decreasing carbon chain length, shorter molecules are perceived more intensely under the equal concentrations used here.  Thus we also added a vapor pressure correction, motivated by the exponential form of the published vapor pressure vs carbon chain length relationship, to account for potential differences in salience across mixture components. 

WE NEED TO WRITE HOW THE MODELS WERE FIT. MIN(RSS?)] Each model was fit to the ?mean? data path by minizing the residuam sum of squares.

Model fit. [HERE WE CAN DESCRIBE THE DATA TRANSFORMATION FOR AGGREGATION ACROSS EXPERIMENT, THEN HOW THE MODEL PARAMETERS WERE FIT, AND FIT STATISTICS, RSS, R^2, AIC, ect. THEY CAN EASILY ALL GO IN A SINGLE TABLE]. 

We found that the nearest neighbor model was the best fitting model (Fig X).  Because vapor pressure increases with decreasing carbon chain length, shorter molecules are perceived more intensely under the equal concentrations used here.  Thus we also added a vapor pressure correction, motivated by the exponential form of the published vapor pressure vs carbon chain length relationship, to account for potential differences in salience across mixture components.  This helped account for asymmetries in the observed generalization data, improving model fit.  