When an animal learns to associate a stimulus $S^+$ with reward, this learning may generalize to other, novel stimuli. 
The greater the physical similarity between the $S^+$ and the novel stimulus, the greater the generalization, and the greater the presumed perceptual similarity. 
However, perceptual relatedness among olfactory stimuli is hard to determine, as physiochemical similarity does not always predict similar behavioral responses. 
Natural odors are typically mixtures of multiple components, which pose a further challenge when estimating their perceptual similarities.  
Similarity between mixtures is presumably determined by similarity of the corresponding components, but it is unclear how this is computed by the olfactory system.  
To answer this, we trained mice to associate a binary reference mixture ($S^+$) with reward, and then tested response probability to other, perceptually similar binary test mixtures.  
Across 5 experiments, we parametrically varied the mixtures in several ways, and assessed generalization as these parameters were varied.  
We then evaluated three candidate approaches for evaluating perceptual similarity, according to their correspondence with the data from the 5 experiments.  
We found the best fit for a model in which, for each component \textit{A} of the $S^+$, only the similarity of the ``nearest'' component \texit{A*} of a test mixture was used to predicting behavioral response, with nearness determined by physiochemical distance.  
We conclude that the olfactory system may deemphasize or discard information about other components not perceptually ``near'' enough to any of those in the $S^+$.