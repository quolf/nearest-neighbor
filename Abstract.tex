\section{Abstract} 

When an animal learns to associate a stimulus $S+$ with reward, this learning may generalize to other, novel stimuli. Typically, the greater the physical similarity between the $S+$ and the novel stimulus, the greater the generalization, and the greater the presumed perceptual similarity. However, perceptual relatedness among olfactory stimuli is hard to determine, as the physiochemical similarity does not always predict similar behavioral responses.

Natural odors are typically mixtures of multiple components, which pose a further challenge when estimating their perceptual similarities.  Similarity can be achieved either by strictly changing the number of overlapping molecular components (OVERLAP) between stimuli, or by manipulating the overall physicochemical similarity of the components across mixtures (SIMILARITY), or both.  We asked how OVERLAP and SIMILARITY contribute to olfactory generalization by training mice to associate a binary reference mixture ($S+$) with reward, and then testing response probability to other, perceptually similar binary test mixtures.  In 4 experiments, we parametrically varied OVERLAP and SIMILARITY for the test mixtures relative to the reference, and found that OVERLAP is the dominant factor, with SIMILARITY of secondary importance.  These results were robust across individuals and strains of mice.

We found a good fit between the data and a model in which, for each component \textit{A} of the $S+$, only the similarity of the ``nearest'' component \texit{A*} of a test mixture was useful for predicting behavioral generalization, with nearness determined by physicochemical distance.  In this model, OVERLAP is a special case of nearness that has outsized influence over generalization.  