\section{Discussion}

The scalar model clearly fails, and did so under all variation in function forms that we could devise.  It is fundamentally unable to account for the fact that mixtures cannot be represented by a single number.  Increasing the carbon chain length of one mixture component cannot be offset by decreasing the carbon chain length of another.  The all-to-all model also fails, apparently because some mixture components are irrelevant in determining generalization.  Giving them equal weight in the calculation results in poor predictions.  The nearest neighbor model is remarkably simple, and works well, suggesting that an important predictor of generalization is the how similar the most similar component of the test mixture is to a given component of the S+.  

This model is consistent with the result that fractional overlap between mixtures, i.e. the fraction of components they share in common, is predictive of mixture similarity in humans (Bushdid et al, 2014).  Indeed, the nearest neighbor model predicts that generalization should scale with fractional overlap, not the absolute number of overlapping components, across mixture sizes (Fig. X), as observed in that report.  The model also works without giving special preference to components that are identical, above and beyond that implied by the difference in carbon chain length.  Whether that generality would be observed across other molecular dimensions, such as changes in functional group, hydrophilicity, or molecular weight, is a question that future experiments can address.   
