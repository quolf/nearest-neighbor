\section{Discussion}
\label{sec:discussion}
We conducted 5 sets of mixture generalization experiments to constrain 3 models for the determination of mixture similarity, showing that the nearest neighbor model had the best fit to the data.  
The scalar model performed poorly, and did so under all variation in functional forms that we could devise (data not shown).  
It is fundamentally unable to account for the fact that odorant mixtures cannot be represented by a single number. 
For example, increasing the carbon chain length of one mixture component cannot be offset by decreasing the carbon chain length of another. 
The all to all model exhibited only slightly poorer fit than nearest neighbor, but involved a more complex computation ($N^2$ components comparisons in all to all vs $N$ for nearest neighbor).  
Thus the additional $N^2 - N$ comparisons did not improve the fit to the data, and in fact made it worse.  
We conclude that, at least for these generalization data, only the $N$ nearest neighbor component comparisons are predictive of mixture similarity.  

Can this model work for data produced in discrimination rather than generalization experiments?  
The nearest neighbor model is consistent with the result from human odorant discrimination that fractional overlap between mixtures, i.e. the fraction of components they share in common, is predictive of mixture similarity \cite{24653035}.  
For example, if we determine the nearest neighbor in mixture B for each component in mixture A, this nearest neighbor will be the component itself if that component is shared, and will be something possibly quite distinct if it is not.  If we assume that the nearest neighbor for unshared components is not particularly "near" in some hypothetical perceptual space, then the nearest neighbor model predicts that perceptual similarity will be proportional to fractional overlap, not with the absolute number of shared components, across mixture sizes, as observed for mixture of size $N=10$, $20$, and $30$ in that report \cite{24653035}.

The models described here do not give special preference to components that are identical, above and beyond that implied by the difference in carbon chain length.  Whether that concept of "molecular feature distance" can work generally across other dimensions, such as changes in functional group, hydrophilicity, or molecular weight, is a question that future experiments can address.  
With a better understanding of the perceptual correlates of those features, it should be possible to build a general nearest neighbor model that can be used to predict mixture similarity for odorant mixtures that contain a wide variety of molecular components.  