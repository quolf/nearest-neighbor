\section*{Discussion}
\label{sec:discussion}

This asymmetry could be an effect of the genetic line or to the perceptual of short chain length odorants...

Describe behavioral results first…..

Explain map to more complex mixtures…

In order to understand how perceptual similarity is determined across mixtures, we conducted five olfactory generalization experiments with binary mixtures of straight-chain aliphatic alcohols.  
In each experiment, we trained mice to associate a single mixture ($S^+$) with liquid reward and then tested their conditioned responses during exposure to the $S^+$ or to related mixtures (\ref{sec:methods}).  
Across experiments, the relationship between the $S+$ test stimuli was varied to determine what features of the $S^+$ were most effectively generalized during testing.  

The \textit{Mean} model clearly fails, and did so under all variation in function forms that we could devise.  It is fundamentally unable to account for the fact that mixtures cannot be represented by a single number.  Increasing the carbon chain length of one mixture component cannot be offset by decreasing the carbon chain length of another.  The \textit{All-to-All} model also fails, apparently because some mixture components are irrelevant in determining generalization.  Giving them equal weight in the calculation results in poor predictions.  The \textit{Nearest-Neighbor} model is remarkably simple, and works well, suggesting that an important predictor of generalization is the how similar the most similar component of the test mixture is to a given component of the S+.  

This model is consistent with the result that fractional overlap between mixtures, i.e. the fraction of components they share in common, is predictive of mixture similarity in humans \cite{24653035}.  Indeed, the \textit{Nearest-Neighbor} model predicts that generalization should scale with fractional overlap, not the absolute number of overlapping components, across mixture sizes (Fig. \ref{fig:entirely}), as observed in that report.  The model also works without giving special preference to components that are identical, above and beyond that implied by the difference in carbon chain length.  Whether that generality would be observed across other molecular dimensions, such as changes in functional group, hydrophilicity, or molecular weight, is a question that future experiments can address.  

One recent study attempted to predict mixture perceptual similarity from molecular structure \cite{24068899}, using angle difference in a high-dimensional molecular feature space.  Their approach was to convert each molecule in a mixture to molecular feature vector, and sum the vectors in a mixture to produce a single mixture vector.  Conceptually, this is similar to our \textit{Mean} model, in that it assigns a single value to each mixture, but in their case it is a vector value.  While we cannot construct an analogous model using only carbon chain length (with that as the only dimension, angle as no meaning), we can use their model with their identified molecular features, ignoring carbon chain length altogether.  The nearest neighbor version of such a model might then be to compute the angle between each molecule in the $S+$ and its nearest (smallest angle in the molecular feature space) neighbor in the test stimulus, and to take the circular mean of those angles.  [[Brian I can do everything speculated about in this paragraph, and have in fact begun it. It would then be in the results section; however, I think there is enough in that idea to make a whole separate paper, in which I am first author]].  