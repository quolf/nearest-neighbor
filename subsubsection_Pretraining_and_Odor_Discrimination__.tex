\subsubsection*{Pretraining and Odor Discrimination}
\label{sec:methods_pretraining}
Pretraining and Odor discrimination training proceeded according to Slotnick and Restrepo (1994). Discrimination training consisted of the standard Go/No Go task in which subjects were trained to lick in the presence of one odor, the $S+$, and not in the presence of another odor, the $S-$.  Each odorant discrimination sessions began with a ``warm up'' block of 10 trials in which all trials were $S+$ in order to establish robust responding to the $S+$  (Nevin, 1998, Shahan 2014).  The remainder of the session consisted of 9 blocks of 10 $S+$ trials and 10 $S-$ trials, ordered pseudo-randomly so that no more than 3 trials of the same type ($S+$ or $S-$) occurred consecutively within a block. Once a subject responded with at least $80\%$ correct for two consecutive sessions, we advanced the protocol to partial reinforcement.