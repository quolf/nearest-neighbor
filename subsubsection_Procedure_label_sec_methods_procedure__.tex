\subsubsection*{Procedure}
\label{sec:methods_procedure}
The odorant discrimination training and generalization testing protocol was a modified version of the Go/No Go task used by Slotnick and colleagues \cite{18428626}. Each experiment consisted of four phases: Pre-training, Odorant discrimination (full reinforcement), Odorant discrimination (partial reinforcement), and Generalization Testing. During pre-training, subjects were trained to lick the liquid delivery tube and keep their head in the odor tube for several seconds. No odor was presented during pre-training. During odorant discrimination (full reinforcement), subjects were trained to lick the reinforcer spout for chocolate milk when one odorant (the $S+$) was presented, but not when mineral oil (the $S-$) was presented; correct responses to all $S+$ trials were reinforced. Because generalization trials with the novel stimuli must be conducted in the absence of reinforcement to prevent training the animal to respond to the novel stimuli, subjects were trained to persist in responding during low rates of reinforcement in Odorant discrimination (partial reinforcement). In that phase, the frequency of reinforcement on $S+$ trials was gradually reduced from $100\%$ to $20\%$. During generalization testing, each subject's responses to three novel, unreinforced odorants were examined.

