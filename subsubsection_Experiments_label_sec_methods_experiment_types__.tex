\subsubsection*{Experiments}
\label{sec:methods_experiment_types}
Five different experiments were conducted across different strains of mice. 
Due to the availability of the mice, MEK2 were only trained and tested with experiments 1-3, while C57 was trained and tested with experiments 1-5.  
Figure \ref{tab:design} outlines the stimuli used as the $S+$ and for generalization testing in each experiment. 
For experiments 1 and 5, the stimuli were pure monomolecular odorants. 
For experiments 2, 3, and 4 all stimuli were binary mixtures that differed in overlap between the $S+$ and the test mixtures. 

\begin{enumerate}[a]
\item \textit{Single-Linear}. Each stimulus was a simple dilution of a 2-n-alcohol (where `n' refers to carbon chain length, e.g. 2-heptanol) to $0.1\%$ concentration by volume in mineral oil.
The generalization stimuli ($G1$, $G2$, $G3$) differed from the $S+$ by either 1 or 2 in carbon chain length, in either direction.  
This experiment tested generalization in the limiting case of stimuli with only one component.   

\item \textit{Double-Linear}. Similar to Single-Linear, except that each stimulus was a mixture of a 2-n-alcohol with the corresponding 2-(n+1)-alcohol, e.g. a mixture of 2-hexanol and 2-heptanol.  
Each component was diluted to $0.05\%$ concentration by volume in mineral oil, so the total volume of alcohol was $0.1\%$ as in Single-Linear.  
This experiment tested the extension of generalization to binary mixtures.  

\item \textit{Double 1-Shared}. Each stimulus was a binary mixture, composed as in Double-Linear. 
However, each of the generalization stimuli shared with the $S+$ one component (2-Heptanol) and differed from the $S+$ by one component.  
This experiment tested the differential contributions to generalization of shared vs distinct components.  

\item \textit{Double 1-Distant}. Each stimulus was a mixture of 2 odorants, diluted as in Double-Linear. However, the generalization stimuli shared no overlapping components with the $S+$, but had identical or similar average carbon chain lengths. 
For example, the $S+$ was a mixture 2-heptanol and 2-nonanol, with an average chain length of 8, while $G1$ was a mixture of 2-Hexanol and 2-decanol, also with an average carbon chain length of 8.  This experiment tested the contribution of average carbon chain length to generalization.  

\item \textit{Vapor Pressure Control}. When the dilution by volume remains constant across the single 2-n alcohols within our series, the partial vapor pressure decreases approximately 3.3 fold with each increment of carbon chain length in this molecular series. 
To ensure that the results in experiments \textit{1}-\textit{4} could not be explained by simple differences in partial vapor pressures, experiment \textit{5} trained 2-Heptanol at $1.0\%$ dilution as the $S+$, and tested 2-Heptanol at $0.01\%$, $0.1\%$ and $5.0\%$ as the generalization stimuli.

\end{enumerate}

  