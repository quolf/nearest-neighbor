\section*{Introduction}
\label{sec:introduction}
  
Behavioral evaluation of how animals perceive odors is essential to helping understand the wealth of information that has been revealed about neural and molecular bases of olfactory processing (REF). 
Animals respond to odors either innately or by modification of behavior through experience of the odor with an appetitive or aversive reward. 
In either case, one experimental approach is to vary some physical quality of a target odor and evaluate how strongly the animal generalizes a response to a new test odor. 
This approach can establish how perceptually similar the test and target odors are to each other. 
In the tests with a series of similar and dissimlar odors, the response typically decreases as the test odor becomes less similar to the target odor. 
This approach has worked well with monomolecular odorants in both insects and some mammals (REFS). 
When conditioned to a target odorant, such as an alcohol of a particular carbon chain length, animals typically respond strongest to the conditioned odorant and less strongly as features such as carbon chain increase or decrease on either side of the conditioned odorant (REF). 
Changes in the functional group (e.g. from alcohol to ketone) or its position on the carbon chain have stronger effects on generalization responses (REF). 
Finally, differential conditioning of two odorants that are close in chain length, where on odorant is reinforced and the other not reinforced, shifts peak responses away from the conditioned odor and in a direction away from the unreinforced odorant (REF). 
These data indicate that molecular features such as functional group and carbon chain length are somehow represented along perceptual dimensions in an animal’s olfactory system. 
That proposition has been also well supported by studies of sensory and early olfactory coding in both insects and mammals (REFS). 
 However, we need more thorough analyses of ‘metrics’ of olfactory stimulus space to approach a fuller understanding of coding dimensions in the peripheral and central nervous system. 
 Natural odors are often mixtures of a few to many different monomolecular components (REF). 
 Natural odors can differ based on the composition and/or the ratios among the individual components. 
 Important next steps in understanding how odors are encoded and acted on will be in investigation of how qualities of mixtures map onto perception. 
 Yet natural mixtures, whether innate or learned, can contain many components in a complex mixture, such as the urine-based odors used for individual recognition of rodents (REF). 
 Even semiochemicals, which elicit strong innate responses,  can be mixtures of a few to several components (REF). 
 Therefore, because of the potential complexity of natural odor mixtures, it will be necessary to build on the knowledge of responses to monomolecular odorants by using a stepwise approach to more complex mixtures, beginning with systematically varying simpler, binary mixtures. 
 For example \cite{19692594} composed binary mixtures that differed in ratios of the two components in a study of behavioral discrimination and peak shift using the honey bee. 
 This study then went on to show how the differences in the mixture, as well as neural plasticity toward the mixtures, could be tracked into early olfactory coding. 
Systematically controlled mixtures will allow for investigation of how animals may use different behavioral strategies in evaluating mixtures. 
The strategies could depend on the nature of the mixture and on the type of reinforcement used to condition a behavioral response. 
Foraging moths, for example, use only a subset of mixture components from the floral odors they approach for food (REF). Use of the mouse for these types of behavioral studies would allow for use of tools developed for genetic mainipulation, electrophysiological recording and optical imaging and stimulation to test the causal relationships between events measured in the peripheral and central NS and behavior. 
However, mice can typically discriminate monomolecular odorants that differ by only a single carbon in carbon chain \cite{18810459}. 
Such ‘steep’ generalization makes it difficult to evaluate graded similarities in the neural representations. 

Here  we develop a new behavioral protocol for the mouse in which behavioral generalization can be more easily manipulated experimentally. 
We start with variation in the carbon chain length of a conditioned monomolecular odorant, which serves as a reference for strong discrimination. 
We then show how arranging different relationships between binary mixtures used for conditioning and testing shapes generalization in specific ways. 
Finally, we show how the data can be explained by a model in which animals may focus on the use of the most salient ‘predictive’ component of a mixture. 