\subsubsection*{Odorant discrimination (partial reinforcement)}
\label{sec:methods_discrimination}
Because generalization trials with the novel stimuli must be conducted in the absence of reinforcement to prevent training the animal to respond to the novel stimuli, subjects were trained to persist in responding during low rates of reinforcement in this phase. Training was identical to odorant discrimination, except the probability of reinforcement on $S+$ trials after the ``warmup'' block was gradually reduced over four sessions. During the first session, $80\%$ of $S+$ trials resulted in reinforcement for correct responding. The following three sessions reduced the frequency of reinforcement on $S+$ trials to $60\%$, $40\%$ and then $20\%$. The next session after $20\%$ reinforcement was Generalization Testing.

