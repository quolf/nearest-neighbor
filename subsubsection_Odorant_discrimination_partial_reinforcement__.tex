\subsubsection*{Odorant discrimination (partial reinforcement)}
\label{sec:methods_discrimination}
Generalization trials with the novel stimuli were conducted in the absence of reinforcement in order to prevent the subjects from learning new stimulus-reward associations to those novel stimuli.  
In order to ensure that subjects would nonetheless persist in responding during the generalization phase, they were trained under low rates of reinforcement during a preceding partial reinforcement phase. 
Training was identical to odorant discrimination, except the probability of reinforcement on $S+$ trials after the ``warmup'' block was gradually reduced over four sessions. During the first session, $80\%$ of $S+$ trials resulted in reinforcement for correct responding. The following three sessions reduced the frequency of reinforcement on $S+$ trials to $60\%$, $40\%$ and then $20\%$. The next session after $20\%$ reinforcement was Generalization Testing.


  
  