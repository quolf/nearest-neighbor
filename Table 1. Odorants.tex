\begin{table}
\label{tab:design}
    \begin{tabular}{ l | c | c | c | c | c }
        Experiment & Strain & \textbf{$S^+$} & G1 & G2 & G3 \\ 
        \hline
        \textit{i.} Single & C57 & \textbf{6} & 5 & 7 & 8 \\ 
         & MEK & \textbf{7} & 5 & 6 & 8 \\ 
         & ChR:ChAT & \textbf{6} & 5 & 7 & 8 \\ 
        \hline
        \textit{ii.} Double & C57 & \textbf{6+7} & 5+6 & 7+8 & 8+9 \\ 
         & MEK & \textbf{7+8} & 5+6 & 6+7 & 8+9 \\ 
        \hline
        \textit{iii.} 1-Shared & C57 & \textbf{6+7} & 5+7 & 7+8 & 7+9 \\ 
         & MEK & \textbf{7+8} & 5+7 & 6+7 & 7+9 \\ 
        \hline
        \textit{iv.} 1-Distant & C57 & \textbf{7+9} & 6+8 & 6+10 & 8+10 \\ 
        \hline
        \textit{v.} Vapor Pressure Control & C57 & \textbf{7(1\%)} & 7(0.01\%) & 7(0.1\%) & 7(5\%) \\ 
    \end{tabular}
    \caption{Odorants used in each experiment (see Fig. \ref{fig:results} for details)}
\end{table}
Note: Each number odorant number refers to the carbon chain length of a 2-n alcohols. E.g., "7" refers to 2-Heptanol, while "7 + 8" refers to a binary mixture of 2-Heptanol and 2-Octanol. In the vapor pressure control, the percentage refers to the dilution; for all other odorants were diluted to {0.1\%} in mineral oil.